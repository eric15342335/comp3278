\documentclass{article}[a4paper,11pt]
\usepackage{amssymb, amsthm, enumitem}
\usepackage[utf8]{inputenc}
\usepackage[margin=0.4in, bottom=0.75in]{geometry}
\usepackage[fleqn]{amsmath}
\usepackage{microtype}
\usepackage{verbatim}
\usepackage{datetime}
\usepackage[pdfusetitle]{hyperref}
\usepackage{bookmark}
\usepackage{array}
\usepackage{booktabs} 
\usepackage{lmodern}
\usepackage{algorithm}
\usepackage{algpseudocode}

% ----- PDF Metadata -----
\hypersetup{
    pdftitle={COMP3278C Assignment 3},
    pdfauthor={Cheng Ho Ming, Eric (3036216734)}, % <<< Placeholder for author
    pdfsubject={Database Management Systems Assignment},
    pdfkeywords={Database, Functional Dependencies, BCNF, Normalization, COMP3278C, HKU},
    bookmarksopen=true, 
    bookmarksnumbered=true 
}

% ----- Title Block -----
\title{COMP3278C Introduction to Database Management Systems \\ Assignment 3: Functional Dependencies and BCNF Decomposition}
\author{Cheng Ho Ming, Eric (3036216734) \\ Section 2C, 2024}
\date{\today \ \currenttime}

\begin{document}
\maketitle

\section*{Question 1}

Given relation $R(A, B, C, D, E)$ and functional dependencies, $F = \{ A \rightarrow B, B \rightarrow C, CA \rightarrow D, C \rightarrow A, D \rightarrow E \}$. Let $\alpha = \{A, C\}$. Compute the closure $\alpha^+$ under $F$.

\vspace{1em} 

\noindent 
\textbf{Answer:}

$\alpha^+ = \{ A, B, C, D, E \}$

%---------------------------------------------------------
\section*{Question 2}

Given a relation $R(A, B, C)$ and the following functional dependencies:
\[ F = \{ A \rightarrow B, \quad C \rightarrow B \}, \]
compute $F^+$, the closure of all functional dependencies in $F$. Your final answer should present a comprehensive list of all functional dependencies that can be derived from F.

\vspace{1em}

\noindent 
\textbf{Answer:}

\begin{center}
\begin{tabular}{@{} l *{7}{p{0.125\textwidth} @{}} } 
\toprule
\textbf{} & \textbf{\{A\}} & \textbf{\{B\}} & \textbf{\{C\}} & \textbf{\{AB\}} & \textbf{\{AC\}} & \textbf{\{BC\}} & \textbf{\{ABC\}} \\
\midrule
\textbf{Attribute} \\ \textbf{set closure} & 
$\{ A, B \}$ & % Closure of A
$\{ B \}$ & % Closure of B
$\{ B, C \}$ & % Closure of C
$\{ A, B \}$ & % Closure of AB
$\{ A, B, C \}$ & % Closure of AC
$\{ B, C \}$ & % Closure of BC
$\{ A, B, C \}$ \\ % Closure of ABC
\midrule[\heavyrulewidth]
\textbf{FD} & & & & & & & \\ &
\multicolumn{1}{p{0.1\textwidth}}{\raggedright
    % List FDs derived from {A}+ here
    $A \rightarrow A$
    $A \rightarrow B$
} &
\multicolumn{1}{p{0.1\textwidth}}{\raggedright
    % List FDs derived from {B}+ here
    $B \rightarrow B$
} &
\multicolumn{1}{p{0.1\textwidth}}{\raggedright
    % List FDs derived from {C}+ here
    $C \rightarrow B$
    $C \rightarrow C$
} &
\multicolumn{1}{p{0.1\textwidth}}{\raggedright
    % List FDs derived from {AB}+ here
    $AB \rightarrow A$
    $AB \rightarrow B$
    $AB \rightarrow AB$
} &
\multicolumn{1}{p{0.1\textwidth}}{\raggedright
    % List FDs derived from {AC}+ here
    $AC \rightarrow A$
    $AC \rightarrow B$
    $AC \rightarrow C$
    $AC \rightarrow AB$
    $AC \rightarrow AC$
    $AC \rightarrow BC$
    $AC \rightarrow ABC$
} &
\multicolumn{1}{p{0.1\textwidth}}{\raggedright
    % List FDs derived from {BC}+ here
    $BC \rightarrow B$
    $BC \rightarrow C$
    $BC \rightarrow BC$    
} &
\multicolumn{1}{p{0.1\textwidth}}{\raggedright
    % List FDs derived from {ABC}+ here
    $ABC \rightarrow A$
    $ABC \rightarrow B$
    $ABC \rightarrow C$
    $ABC \rightarrow AB$
    $ABC \rightarrow AC$
    $ABC \rightarrow BC$
    $ABC \rightarrow ABC$
} \\
\bottomrule
\end{tabular}
\end{center}

%---------------------------------------------------------
\newpage 
\section*{Question 3}

Consider the relation schema:
\[ R = (A, B, C, D, E) \]
with the following functional dependencies:
\[ F = \{ A \rightarrow B, \quad C \rightarrow D, \quad AB \rightarrow E, \quad C \rightarrow E, \quad AB \rightarrow C \}. \]

\vspace{1em}

\subsection*{(1) Identify all candidate keys for $R$.}

\noindent 
\textbf{Steps:}

% Provide your step-by-step derivation here.
\begin{enumerate}
    \item Is A a superkey?
    \begin{itemize}
        \item $\{A\}^+ = \{A, B, C, D, E\} = R$
        \item $A$ is a superkey.
    \end{itemize}
    \item Is B a superkey?
    \begin{itemize}
        \item $\{B\}^+ = \{B\} \neq R$
        \item $B$ is \textbf{NOT} a superkey.
    \end{itemize}
    \item Is C a superkey?
    \begin{itemize}
        \item $\{C\}^+ = \{C,D,E\} \neq R$
        \item $C$ is \textbf{NOT} a superkey.
    \end{itemize}
    \item Is D a superkey?
    \begin{itemize}
        \item $\{D\}^+ = \{D\} \neq R$
        \item $D$ is \textbf{NOT} a superkey.
    \end{itemize}
    \item Is E a superkey?
    \begin{itemize}
        \item $\{E\}^+ = \{E\} \neq R$
        \item $E$ is \textbf{NOT} a superkey.
    \end{itemize}
    \item (Skipped AB, AC, AD, AE since A is a superkey)
    \item Is BC a superkey?
    \begin{itemize}
        \item $\{BC\}^+ = \{B,C,D,E\} \neq R$
        \item $BC$ is \textbf{NOT} a superkey.
    \end{itemize}
    \item Is BD a superkey?
    \begin{itemize}
        \item $\{BD\}^+ = \{B,D\} \neq R$
        \item $BD$ is \textbf{NOT} a superkey.
    \end{itemize}
    \item Is BE a superkey?
    \begin{itemize}
        \item $\{BE\}^+ = \{B,E\} \neq R$
        \item $BE$ is \textbf{NOT} a superkey.
    \end{itemize}
    \item Is CD a superkey?
    \begin{itemize}
        \item $\{CD\}^+ = \{C,D,E\} \neq R$
        \item $CD$ is \textbf{NOT} a superkey.
    \end{itemize}
    \item Is CE a superkey?
    \begin{itemize}
        \item $\{CE\}^+ = \{C,D,E\} \neq R$
        \item $CE$ is \textbf{NOT} a superkey.
    \end{itemize}
    \item Is DE a superkey?
    \begin{itemize}
        \item $\{DE\}^+ = \{D,E\} \neq R$
        \item $DE$ is \textbf{NOT} a superkey.
    \end{itemize}
    \item (Skipped ABC, ABD, ABE, ACD, ACE, ADE since A is a superkey)
    \item Is BCD a superkey?
    \begin{itemize}
        \item $\{BCD\}^+ = \{B,C,D,E\} \neq R$
        \item $BCD$ is \textbf{NOT} a superkey.
    \end{itemize}
    \item Is BCE a superkey?
    \begin{itemize}
        \item $\{BCE\}^+ = \{B,C,D,E\} \neq R$
        \item $BCE$ is \textbf{NOT} a superkey.
    \end{itemize}
    \item Is BDE a superkey?
    \begin{itemize}
        \item $\{BDE\}^+ = \{B,D,E\} \neq R$
        \item $BDE$ is \textbf{NOT} a superkey.
    \end{itemize}
    \item Is CDE a superkey?
    \begin{itemize}
        \item $\{CDE\}^+ = \{C,D,E\} \neq R$
        \item $CDE$ is \textbf{NOT} a superkey.
    \end{itemize}
    \item (Skipped ABCD, ABCE, ABDE, ACDE since A is a superkey)
    \item Is BCDE a superkey?
    \begin{itemize}
        \item $\{BCDE\}^+ = \{B,C,D,E\} \neq R$
        \item $BCDE$ is \textbf{NOT} a superkey.
    \end{itemize}
    \item (Skipped ABCDE since A is a superkey)
\end{enumerate}

\vspace{1em}
\noindent 
\textbf{Candidate Keys:} $\{ A \}$

\vspace{2em} 

\subsection*{(2) Determine whether $R$ is in Boyce-Codd Normal Form (BCNF).}

R is \textbf{NOT} in BCNF.

\vspace{1em}

% Explain reasoning based on definition of BCNF and the candidate keys found.
BCNF requires that for every non-trivial FD $\alpha \rightarrow \beta$, $\alpha$ must be a superkey.

Candidate key found: $\{ A \}$

To check for violations of BCNF, we only have to check the FDs in $F$.

\begin{itemize}
    \item $A \rightarrow B$: $\{ A \}$ is a superkey. (OK)
    \item $C \rightarrow D$: $\{ C \}$ is not a superkey. (Violation)
    \item $AB \rightarrow E$: $\{ AB \}$ is a superkey since $A$ is a superkey. (OK)
    \item $C \rightarrow E$: $\{ C \}$ is not a superkey. (Violation)
    \item $AB \rightarrow C$: $\{ AB \}$ is a superkey since $A$ is a superkey. (OK)
\end{itemize}

\subsection*{(3) BCNF Decomposition.}

\textbf{Decomposition Process:}

\underline{Iteration 1:}
\begin{itemize}
    \item Start with relation: $R(A,B,C,D,E)$ with $F = \{ A \rightarrow B, C \rightarrow D, AB \rightarrow E, C \rightarrow E, AB \rightarrow C \}$.
    \item Choose a violating FD: $C \rightarrow D$.
    \item Decompose $R$ into:
    \begin{itemize}
        \item $R_1(C,D)$
        \item $R_2(A,B,C,E)$
    \end{itemize}
    \item Check for BCNF:
    \begin{itemize}
        \item $F_1$: $\{ C \rightarrow D, trivials\}$, since $\{C\}^+ = \{C,D\} = R_1$, $R_1$ is in BCNF.
        \item $F_2$: $\{ A \rightarrow B, C \rightarrow E, AB \rightarrow C, AB \rightarrow E, trivials\}$, check each FD:
        \begin{itemize}
            \item $A \rightarrow B$: $\{ A \}^+ = \{ A, B, C, E \} = R_2$. (OK)
            \item $C \rightarrow E$: $\{ C \}^+ = \{ C, E \} \neq R_2$. (Violation)
            \item $AB \rightarrow C$: $\{ AB \}^+ = \{ A, B, C, E \} = R_2$. (OK)
            \item $AB \rightarrow E$: $\{ AB \}^+ = \{ A, B, C, E \} = R_2$. (OK)
            \item $R_2$ is \textbf{NOT} in BCNF.
        \end{itemize}
    \end{itemize}
    \item Check for dependency preserving:
    \begin{itemize}
        \item Since $(F_1 \cup F_2)^+ = F^+$, this decomposition is dependency preserving.
    \end{itemize}
    \item Check for loseless join:
    \begin{itemize}
        \item $(schema (R_1) \cap schema(R_2) \rightarrow schema(R_1)) \Leftrightarrow (C \rightarrow CD)$, where $C \rightarrow CD$ is true.
        \item This decomposition is loseless join.
    \end{itemize}
\end{itemize}

\underline{Iteration 2:}
\begin{itemize}
    \item Start with relation: $R_2(A,B,C,E)$ with $F_2 = \{ A \rightarrow B, C \rightarrow E, AB \rightarrow C, AB \rightarrow E, trivials\}$.
    \item Choose a violating FD: $C \rightarrow E$.
    \item Decompose $R_2$ into:
    \begin{itemize}
        \item $R_3(C,E)$
        \item $R_4(A,B,C)$
    \end{itemize}
    \item Check for BCNF:
    \begin{itemize}
        \item $F_3$: $\{ C \rightarrow E, trivials\}$, since $\{C\}^+ = \{C,E\} = R_3$, $R_3$ is in BCNF.
        \item $F_4$: $\{ A \rightarrow B, AB \rightarrow C, trivials\}$, check each FD:
        \begin{itemize}
            \item $A \rightarrow B$: $\{ A \}^+ = \{ A, B, C \} = R_4$. (OK)
            \item $AB \rightarrow C$: $\{ AB \}^+ = \{ A, B, C \} = R_4$. (OK)
            \item $R_4$ is in BCNF.
        \end{itemize}
    \end{itemize}
    \item Check for dependency preserving:
    \begin{itemize}
        \item $(F_1 \cup F_3 \cup F_4)^+ = F^+$, this decomposition is dependency preserving.
    \end{itemize}
    \item Check for loseless join:
    \begin{itemize}
        \item $(schema (R_3) \cap schema(R_4) \rightarrow schema(R_3)) \Leftrightarrow (C \rightarrow CE)$, where $C \rightarrow CE$ is true.
        \item This decomposition is loseless join.
    \end{itemize}
\end{itemize}

\underline{Result of a BCNF decomposition on $R$:}

\begin{itemize}
    \item $R_1(C,D)$
    \item $R_3(C,E)$
    \item $R_4(A,B,C)$
    \item $F_1 = \{ C \rightarrow D, trivials\}$
    \item $F_3 = \{ C \rightarrow E, trivials\}$
    \item $F_4 = \{ A \rightarrow B, AB \rightarrow C, trivials\}$
\end{itemize}

\end{document}
